{\color{blue} comments by Alessandro}
Following Larin, we assume that it is possible to reabsorb the spurious terms generated by
the prescritpion for $\gamma_5$ with a counterterm, $Z_5$, which can be computed in perturbation
theory, by imposing the validity of the Ward identities for the axial-vector current of the
$Z-f-\bar f$ vertex.
\be
Z_{5,f}=1 +
\alpha \delta Z_{5,f}^{(\alpha)} +
\alpha_s \delta Z_{5,f}^{(\alpha_s)} +
\alpha \alpha_s\delta Z_{5,f}^{(\alpha\alpha_s)} +
\alpha_s^2 \delta Z_{5,f}^{(\alpha_s^2)} + \dots
\ee
It is a finite correction, stemming from the product of a prescription-dependent term in the traces of the Dirac matrices with the poles of the loop integrals.
We insert $Z_{5,f}$, starting from the lowest-order, in all the tree-level vertices involving a $Z$ boson
and expand perturbatively the resulting amplitude.
We are thus interested in the $Z_5$ insertions on the tree-level and 1-loop amplitudes, because at 2-loop level the additional contributions are at 3-loop and higher order.
The amplitude, after these insertions, should be by construction free of spurious terms and each sub-amplitude should satisfy the relevant Ward identities.

We construct the IR subtraction term by adding to each universal divergent term
an additional finite correction which takes into account the prescription adopted to compute
either the Born either the 1-loop QCD or EW amplitudes.
In this way, also the IR subtraction term should be, by construction, free of spurious terms.

We systematically check where the $\delta Z_{5,f}$ factors appear in both squared matrix element and IR subtraction term, looking for a pattern of cancellation of all the $Z_5$ factors, when we compute the subtracted cross section.
