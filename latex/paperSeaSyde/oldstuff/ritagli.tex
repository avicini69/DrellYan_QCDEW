
As a representative example, the production of a pair of large transverse momentum leptons, known as Drell-Yan (DY) process, has been recently studied including the next-to-next-to-leading order (NNLO) mixed strong electroweak (EW) corrections and the next-to-next-to-next-to-leading order (NNNLO) strong corrections.


On the other hand, when we rescale by complex masses, all the singular points
are located away from the real axis
and the decision about the orientation of the cuts could become problematic.
To solve this problem,
we rely on the real-mass rescaling
for the identification of the two sets of points.
We then assign a cut parallel to the real axis
to the points of the real-vauled singularities set.
The imaginary component of the cut, induced by the rescaling by a complex mass,
is an important input when we decide the
trajectory of the solution in the complex plane
from the BCs point to the value of interest.
If we can move in the complex plane
avoiding a crossing of all the cuts parallel to the real axis,
then the precise trajectory must only take care of the position
of the other singularities,
those which are from the very beginning in the complex plane.




For the latter,
it is equivalent whether we rescale the invariants
by a real or a complex mass.







% ************** Used Packages **************

%\RequirePackage{ifpdf} 
%\usepackage{amsmath} 
%\usepackage{mathtools}

%\usepackage{jheppub}
%\usepackage{pstricks}
%\usepackage[final]{pdfpages} 
%\usepackage{ifpdf} 
%\usepackage{slashed}
%\usepackage{ulem}
%\usepackage{hyperref}

%\usepackage{color} 
%\usepackage{graphics}

%\usepackage{etoolbox} 
%\usepackage{fixmath}

%\usepackage{Style/notoccite} 
% \usepackage{caption} 
% \usepackage{subcaption} 
%\usepackage{amsfonts}

%\usepackage{multirow}
%\usepackage{epstopdf}
%\usepackage[compat=1.1.0]{tikz-feynman}
%\usepackage{environ}
%\NewEnviron{eq}{%
%\begin{equation}\begin{split}
%  \BODY
%\end{split}\end{equation}
%}

% For making math symbol large: \mathlarger
%\usepackage{relsize}
%\usepackage{cancel}
%\usepackage{float}
%\usepackage[normalem]{ulem}
%\usepackage{framed}

%% %\usepackage[utf8]{inputenc}
%% %\usepackage[normalem]{ulem}

%% \usepackage{tikz}
%% \usetikzlibrary{positioning,arrows}
%% \usetikzlibrary{decorations.pathmorphing}
%% \usetikzlibrary{decorations.markings}
%% \usetikzlibrary{shapes.geometric}
%% \usepackage{endnotes}
%% %\usepackage{pgflibraryarrows}
%% %\usepackage{pgflibrarysnakes}
%% \tikzset{
%% 	% >=stealth', %%  Uncomment for more conventional arrows
%%     vector/.style={decorate, decoration={snake}, draw},
%%     provector/.style={decorate, decoration={snake,amplitude=2.5pt}, draw},
%%     antivector/.style={decorate, decoration={snake,amplitude=-2.5pt}, draw},
%%     fermion/.style={draw=black,
%%       postaction={decorate},decoration={markings,mark=at position .55
%%         with {\arrow[draw=black]{>}}}}, 
%%     fermionbar/.style={draw=black, postaction={decorate},
%%                        decoration={markings,mark=at position .55 with {\arrow[draw=black]{<}}}},
%%     fermionnoarrow/.style={draw=black},
%%     gluon/.style={decorate, draw=black,decoration={coil,amplitude=4pt, segment length=6pt}},
%%     scalar/.style={dashed,draw=black,
%%       postaction={decorate},decoration={markings,mark=at position .55
%%         with {\arrow[draw=black]{>}}}}, 
%%     scalarbar/.style={dashed,draw=black,
%%       postaction={decorate},decoration={markings,mark=at position .55
%%         with {\arrow[draw=black]{<}}}}, 
%%     scalarnoarrow/.style={dashed,draw=black},
%%     electron/.style={draw=black,
%%       postaction={decorate},decoration={markings,mark=at position .55
%%         with {\arrow[draw=black]{>}}}}, 
%%     bigvector/.style={decorate, decoration={snake,amplitude=4pt}, draw},
%% } 

%% \newcommand{\dd}{\draw}
%% % \newcommand{\nn}{\node}
%% \newcommand{\Poincare}{Poincar\'e\xspace}

% \newcommand{\ep}{\epsilon}

% \newcommand{\DD}{{\cal D}}
% \newcommand{\nn}{\nonumber\\&}
% \newcommand{\D2}{${\cal D}_2$}
% \newcommand{\D3}{${\cal D}_3$}

%% \newcommand{\zb}{\bar{z}}
%% \newcommand{\zbb}{\bar{\bar{z}}}
%% \newcommand{\zbf}{\hat{\bar{z}}}
%% \newcommand{\zt}{\tilde{z}}
%% \newcommand{\ztt}{\tilde{\tilde{z}}}
%% \newcommand{\ztf}{\hat{\tilde{z}}}

%% \newcommand{\xxm}{\bar{t}}
%% \newcommand{\xxp}{\tilde{t}}
%% \newcommand{\xxt}{{\check{t}}}
%% \newcommand{\xxh}{\hat{\tilde{t}}}

%% \newcommand{\xyb}{\bar{w}}
%% \newcommand{\xyt}{\tilde{w}}

%% \newcommand{\rb}{\bar{\rho}}
%% \newcommand{\rrt}{\hat{\rho}}

%% \newcommand{\itwo}{i_2}
%% \newcommand{\mione}{i_1}

%% \newcommand{\GI}{G_I}
%% \newcommand{\sw}{s_{\scriptscriptstyle W}}
%% \newcommand{\cw}{c_{\scriptscriptstyle W}}




%% \newcommand{\smallw}{{\scriptscriptstyle W}}
%% \newcommand{\mt}{m_t} 
%% \newcommand{\mw}{m_\smallw} 
%% \newcommand{\mwsq}{m_\smallw^2} 
%% \newcommand{\mwc}{m_{\smallw 0}}
%% \newcommand{\smallz}{{\scriptscriptstyle Z}}
%% \newcommand{\mz}{m_\smallz} 
%% \newcommand{\mzsq}{m_\smallz^2} 
%% \newcommand{\mzc}{m_{\smallz 0}} 
%% \newcommand{\oa}{${\cal O}(\alpha)$\,} 
%% \newcommand{\oas}{${\cal O}(\alpha_s)$\,} 
%% \newcommand{\oasas}{${\cal O}(\alpha_s^2)$\,} 
%% \newcommand{\oaa}{${\cal O}(\alpha^2)$\,} 
%% \newcommand{\oaas}{${\cal O}(\alpha\alpha_s)$\,} 
%% \newcommand{\sineffl}{\sin\theta_{eff}^{\ell}\,}
%% \newcommand{\coseffl}{\cos\theta_{eff}^{\ell}\,}
%% \newcommand{\seffl}{\sin^2\theta_{eff}^{\ell}\,}
%% \newcommand{\ceffl}{\cos^2\theta_{eff}^{\ell}\,}
%% \newcommand{\eps}{\varepsilon} 
%% \newcommand{\muf}{\mu_F}
%% \newcommand{\mur}{\mu_R}
%% \newcommand{\sinw}{\sin\theta_W}
%% \newcommand{\cosw}{\cos\theta_W}
%% \newcommand{\swtwo}{\sin^2\theta_W}
%% \newcommand{\swfour}{\sin^4\theta_W}
%% \newcommand{\cwtwo}{\cos^2\theta_W}

%\newcommand{\av}[1]{{\color{red} {#1}}}
%\newcommand{\nr}[1]{{\color{blue} #1}}

