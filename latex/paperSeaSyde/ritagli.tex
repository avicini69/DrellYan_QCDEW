%The system simplifies further, if it is possible, by an appropriate change of basis, to cast it in the following differential form \cite{Henn:2013pwa}
%\be
%d\vec{J}=\varepsilon\, d\mathbf{\tilde A}\,\vec{J},\quad\quad
%\mathbf{\tilde A}=\sum_l \mathbf{\tilde A}_l \log l \, ,
%\label{can}
%\ee
%where $l$ are the letters, i.e. the combination of kinematic variables which parametrise the singular structure of the scattering amplitude and in particular the various internal thresholds and pseudo-thresholds of the Feynman integral. 
%The structure of equation \ref{can} is called $\varepsilon$-form or canonical form.
%A further important simplification of the problem is achieved if the choice of the MIs allows to cast the system of equations in canonical form \cite{Henn:2013pwa}. In this case the system becomes
%Now the $\varepsilon$ dependence of the matrices $\mathbf{A}$ comes out to be completely factorised. In this case it is customary to look for solutions in power series of $\varepsilon$, $\vec{J}=\sum_{k=0}^{\infty} \varepsilon^k \, \vec{J}^{(k)}$, re-scaling appropriately by a suitable power of $\varepsilon$ the masters when moving from the basis $\vec{I}$ of Eq.~(\ref{eq:system}) to the basis $\vec{J}$ of (\ref{can}) \cite{Henn:2014qga}.
