%%
%% This is file `cimsmple.tex'
%%
%% 
%% IMPORTANT NOTICE:
%% 
%% For the copyright see the source file.
%% 
%% Any modified versions of this file must be renamed
%% with new filenames distinct from cimsmple.tex.
%% 
%% 
%% This generated file may be distributed as long as the
%% original source files, as listed above, are part of the
%% same distribution. (The sources need not necessarily be
%% in the same archive or directory.)
%%%%%%%%%%%%%%%%%%%%%%%%%%%%%%%%%%%%%%%%%%%%%%%%%%
%%%%%%%%%%%%%%%%%%%%%%%%%%%%%%%%%%%%%%%%%%%%%%%%%%
%%%%%%%%%%%%%%%%%%%%%%%%%%%%%%%%%%%%%%%%%%%%%%%%%%

\documentclass{cimento}

%%%%%%%%%%%%%
%
%VERY IMPORTANT
% 
% If you are preparing Enrico Fermi School of
% Physics report, please read the bundled file README.varenna 
%
%%%%%%%%%%%%


%%%%%%%%%%%%%%%%
%
% VERY IMPORTANT 
%
% In order to set a Copyright owner please use and fulfill the following command 
%\setcopyright{CERN on behalf the XXXXX Collaboration}
%
%
%%%%%%%%%%%%%%%

%\usepackage{graphicx}  % got figures? uncomment this

\title{Sample paper for the cimento class}
\author{S.~Summers\from{ins:x}\ETC,
J.~Grey\from{ins:x},
H.~Smith\from{ins:x}
        \atque
T.~Moore\from{ins:y}\thanks{Any footnote to author.}}
\instlist{\inst{ins:x} INFN, Sezione di Bologna - Bologna, Italy
  \inst{ins:y} Dipartimento di Fisica, Universit\`a di Roma - Roma, Italy}
%% When only one author is present, please do not use the command \from{} near the author name.





\begin{document}

\maketitle

\begin{abstract}
This sample paper is intended to briefly expose the differences
between a standard \LaTeX\ article and a paper based upon the
\texttt{cimento} class. References and equations are not allowed here. 
\end{abstract}

\section{Description}
This is a very short sample paper distributed with the class
\texttt{cimento}.
It is just a collection of examples about the syntax of commands
which behave in a different way from the standard \LaTeX
and/or new commands not defined in \LaTeX.


%\begin{figure}
%\includegraphics{foo}     % includes figure foo.eps
%\caption{Description of the figure.}
%\end{figure}


This sample is not meant to provide the complete documentation for the class.
You can also use this file as a template for your own paper:
copy it to another filename and then modify as needed.

\section{Examples}

\subsection{Tables}
Table~\ref{tab:pricesI}
inserted at this point.

\begin{table}
  \caption{Prices of important items.}
  \label{tab:pricesI}
  \begin{tabular}{rcl}
    \hline
      Item 1      & 1500  & EUR    \\
      Item 2 & 15000 & EUR    \\
      Item 3      & 1500  & dollars \\
    \hline
      Item 4     & .25   & dollars \\
      Item 5         & 1.25  & dollars \\
      Item 6         & 1     & dollars \\
    \hline
  \end{tabular}
\end{table}





\subsection{Mathematics}
Here is a lettered array~(\ref{e.all}), with eqs.~(\ref{e.house})
and~(\ref{e.phi}):
\begin{eqnletter}
 \label{e.all}
 \drm x_\sy{F} & = & 1.2\cdot10^3\un{cm}, \qquad
                     \tx{where\ } \sy{F} = \tx{Fermi}    \label{e.house}\\
 \phi_i        & = & i\pi                                \label{e.phi}
\end{eqnletter}

\subsection{Citations}
We're almost done, just some citations~\cite{ref:apo}
and we will be over~\cite{ref:pul,ref:bra}.
 


\acknowledgments
The author acknowledge XXX, YYY.

\begin{thebibliography}{0}
\bibitem{ref:apo} \BY{Einstein A. \atque Fermi E.}
  \IN{Phys. Rev. A}{13}{1999}{12};
  \SAME{69}{999}{1666}.
\bibitem{ref:pul} \BY{Newton I.}
  preprint INFN 8181.
\bibitem{ref:bra} \BY{Bragg~B.}
  \TITLE{Complete Works}, in \TITLE{Workers Playtime}, edited by \NAME{Tizio A. \atque Caio B.} (Unexeditor, Bologna) 1997, pp.~1-10.



\end{thebibliography}

\end{document}
%%
%% End of file `cimsmple.tex'.
